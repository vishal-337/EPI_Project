%% This is file `sample-sigconf.tex',
%% generated with the docstrip utility.
%%
%% The original source files were:
%%
%% samples.dtx  (with options: `all,proceedings,bibtex,sigconf')
%% 
%% IMPORTANT NOTICE:
%% 
%% For the copyright see the source file.
%% 
%% Any modified versions of this file must be renamed
%% with new filenames distinct from sample-sigconf.tex.
%% 
%% For distribution of the original source see the terms
%% for copying and modification in the file samples.dtx.
%% 
%% This generated file may be distributed as long as the
%% original source files, as listed above, are part of the
%% same distribution. (The sources need not necessarily be
%% in the same archive or directory.)
%%
%%
%% Commands for TeXCount
%TC:macro \cite [option:text,text]
%TC:macro \citep [option:text,text]
%TC:macro \citet [option:text,text]
%TC:envir table 0 1
%TC:envir table* 0 1
%TC:envir tabular [ignore] word
%TC:envir displaymath 0 word
%TC:envir math 0 word
%TC:envir comment 0 0
%%
%% The first command in your LaTeX source must be the \documentclass
%% command.
%%
%% For submission and review of your manuscript please change the
%% command to \documentclass[manuscript, screen, review]{acmart}.
%%
%% When submitting camera ready or to TAPS, please change the command
%% to \documentclass[sigconf]{acmart} or whichever template is required
%% for your publication.
%%
%%


\documentclass[sigconf]{acmart}
%%
%% \BibTeX command to typeset BibTeX logo in the docs
\settopmatter{printacmref=false}
\AtBeginDocument{%
  \providecommand\BibTeX{{%
    Bib\TeX}}}

%% Rights management information.  This information is sent to you
%% when you complete the rights form.  These commands have SAMPLE
%% values in them; it is your responsibility as an author to replace
%% the commands and values with those provided to you when you
%% complete the rights form.
\setcopyright{none}      % remove ACM rights/permissions text
\settopmatter{printacmref=false} % hide "ACM Reference Format"
\acmDOI{}                % clear DOI line
\acmISBN{}               % clear ISBN line
\acmPrice{}              % clear price line (if any)

%% These commands are for a PROCEEDINGS abstract or paper.
\acmConference[]{}{June 03--05,
  2018}{Woodstock, NY}
%%
%%  Uncomment \acmBooktitle if the title of the proceedings is different
%%  from ``Proceedings of ...''!
%%
%%\acmBooktitle{Woodstock '18: ACM Symposium on Neural Gaze Detection,
%%  June 03--05, 2018, Woodstock, NY}
\acmISBN{978-1-4503-XXXX-X/2018/06}


%%
%% Submission ID.
%% Use this when submitting an article to a sponsored event. You'll
%% receive a unique submission ID from the organizers
%% of the event, and this ID should be used as the parameter to this command.
%%\acmSubmissionID{123-A56-BU3}

%%
%% For managing citations, it is recommended to use bibliography
%% files in BibTeX format.
%%
%% You can then either use BibTeX with the ACM-Reference-Format style,
%% or BibLaTeX with the acmnumeric or acmauthoryear sytles, that include
%% support for advanced citation of software artefact from the
%% biblatex-software package, also separately available on CTAN.
%%
%% Look at the sample-*-biblatex.tex files for templates showcasing
%% the biblatex styles.
%%

%%
%% The majority of ACM publications use numbered citations and
%% references.  The command \citestyle{authoryear} switches to the
%% "author year" style.
%%
%% If you are preparing content for an event
%% sponsored by ACM SIGGRAPH, you must use the "author year" style of
%% citations and references.
%% Uncommenting
%% the next command will enable that style.
%%\citestyle{acmauthoryear}


%%
%% end of the preamble, start of the body of the document source.
\begin{document}

%%
%% The "title" command has an optional parameter,
%% allowing the author to define a "short title" to be used in page headers.
\title{Opioid Epidemic Network Modeling}

%%
%% The "author" command and its associated commands are used to define
%% the authors and their affiliations.
%% Of note is the shared affiliation of the first two authors, and the
%% "authornote" and "authornotemark" commands
%% used to denote shared contribution to the research.




\author{Vishal Maradana}
\affiliation{%
  \institution{Georgia Institute of Technology}
  \city{Atlanta,GA}
  \country{United States}}
\email{vmaradana3@gatech.edu}

\author{Rishi Bandi}
\affiliation{%
  \institution{Georgia Institute of Technology}
  \city{Atlanta,GA}
  \country{United States}}
\email{rbandi6@gatech.edu}

%%
%% By default, the full list of authors will be used in the page
%% headers. Often, this list is too long, and will overlap
%% other information printed in the page headers. This command allows
%% the author to define a more concise list
%% of authors' names for this purpose.
\renewcommand{\shortauthors}{Trovato et al.}

%%
%% The abstract is a short summary of the work to be presented in the
%% article.
\begin{abstract}
This project models high-rate opioid prescribing as a contagious behavior diffusing across U.S. states through geographic and professional proximity. Using CDC dispensing data (2006--2018), we infer a directed influence network among states and apply spatial autoregressive models to predict prescribing rates. Our state-level model achieves $R^2 = 0.884$, with 98.1\% self-persistence and a small but significant 0.9\% neighbor spillover effect. Tennessee, Nevada, and South Carolina emerge as top influencers. At the county level, we identify 345 ``superspreader'' counties that adopted high prescribing before their states, with Texas counties dominating intra-state influence rankings. The county model ($R^2 = 0.792$) reveals stronger spatial effects (5\%) than the state model, suggesting finer-scale geographic dynamics. These findings support regionally targeted interventions over state-by-state approaches.

\end{abstract}

%%
%% The code below is generated by the tool at http://dl.acm.org/ccs.cfm.
%% Please copy and paste the code instead of the example below.
%%





%%
%% Keywords. The author(s) should pick words that accurately describe
%% the work being presented. Separate the keywords with commas.

%% A "teaser" image appears between the author and affiliation
%% information and the body of the document, and typically spans the
%% page.
\begin{teaserfigure}

  \includegraphics[width=\textwidth]{opiod.png}
  \caption{Seattle Mariners at Spring Training, 2010.}
  \Description{Enjoying the baseball game from the third-base
  seats. Ichiro Suzuki preparing to bat.}
  \label{fig:teaser}
\end{teaserfigure}

\received{20 February 2007}
\received[revised]{12 March 2009}
\received[accepted]{5 June 2009}

%%
%% This command processes the author and affiliation and title
%% information and builds the first part of the formatted document.
\maketitle

\section{Introduction}
The opioid crisis remains one of the most devastating public health emergencies in United States history. Since the late 1990s, opioid-related overdose deaths have claimed over 500,000 lives, with prescription opioids playing a central role in initiating and sustaining the epidemic. While significant attention has focused on individual prescribing decisions, clinical guidelines, and patient-level risk factors, less is understood about how prescribing practices spread across geographic boundaries. A growing body of research suggests that medical behaviors do not emerge in isolation, they propagate through professional networks, institutional norms, regional practice patterns, and policy environments. This raises an important question: can high-rate opioid prescribing be modeled as a contagious behavior that diffuses spatially across states and counties?

Understanding prescribing as a spatial phenomenon has significant implications for intervention design. If high prescribing rates in one region systematically precede increases in neighboring regions, then targeted interventions in key ``source'' areas could produce outsized effects. Conversely, if prescribing patterns are primarily driven by local factors with minimal cross-boundary influence, then interventions should focus on state-specific or county-specific determinants. Distinguishing between these scenarios requires methods that can both detect spatial dependencies and quantify their predictive importance.

This study treats opioid prescribing as a potential spatial contagion, examining whether high-rate prescribing diffuses across U.S. states and counties through geographic adjacency. Using CDC dispensing data spanning 2006 to 2018, we pursue two complementary objectives. First, we construct directed influence networks at both state and county levels to identify which regions historically preceded others in adopting high-prescribing behavior. By tracking year-over-year transitions into ``high'' status, we infer potential influence relationships and rank regions by their centrality in the resulting network. Second, we develop spatial autoregressive models that predict future prescribing rates based on both a region's own historical rate and the average rate of its geographic neighbors. This modeling approach allows us to decompose prescribing dynamics into self-persistence (how much a region's past predicts its future) versus spatial spillover (how much neighbors' rates contribute).

By conducting parallel analyses at the state and county levels, we aim to understand whether spatial effects operate differently at different geographic scales. Such multi-scale insights could inform whether public health interventions should target regional clusters, individual high-influence states, or specific ``superspreader'' counties that adopt risky prescribing norms before their surrounding areas.






\section{Related Work}
Work on diffusion and influence in networks provides the algorithmic backbone for our project. Kempe, Kleinberg, and Tardos \cite{Kempe2003KDD} formalized two canonical views of spread—the Independent Cascade (IC) and Linear Threshold (LT) models—demonstrating that influence maximization is submodular and admits a tight greedy approximation. Their framework clarifies why heuristic targeting (e.g., high degree) often fails by ignoring redundancy. However, their approach assumes a known network and focuses on forward optimization (who to seed), whereas our challenge is the inverse problem: inferring the underlying influence structure from observed adoption times.

To address this inverse problem, Gomez-Rodriguez et al. developed NETINF \cite{GomezRodriguez2010NetInf}, which infers directed edges from cascade timing data using a submodular optimization approach. While NETINF effectively recovers network structures in large-scale web data, its reliance on tree-structured cascades and multiple independent contagion events does not align perfectly with the single, cyclical annual stream of state-level opioid data. Consequently, we adopt their timing-to-edges intuition—using adoption order to reveal hidden edges—but employ a transparent, count-based construction method better suited for a single macroscopic trajectory.

In the domain of public health networks, Kaminski et al. \cite{Kaminski2023HRJ} constructed a provider network in Indiana based on shared patients, revealing that data-driven clinical communities often diverge from administrative districts. This highlights the limitations of purely geographic targeting and suggests that influence flows through professional relationships. Similarly, O’Malley et al. \cite{Ran2024ANS} quantified risky prescribing within shared-patient networks, finding that homophily exists at the supra-dyadic (triadic) level. Their findings suggest that local norms and clustered structures help sustain behavior, cautioning against interventions that focus solely on single “hubs.” However, both studies focus on clinician-level snapshots without explicit counterfactual policy simulations. Our work scales these insights to the state level to analyze inter-jurisdictional influence over nearly two decades.

Recent work by Yang et al. \cite{Yang2022JBI} further expanded this field by demonstrating that bipartite-aware centrality measures outperform traditional algorithms in detecting high-risk prescribing patterns. Collectively, this literature suggests a coherent path: we leverage the diffusion logic of \cite{Kempe2003KDD} and the inference perspective of \cite{GomezRodriguez2010NetInf}, but adapt them to the constraints of aggregate health data. By bridging the structural insights of \cite{Kaminski2023HRJ} and \cite{Ran2024ANS} with policy timelines, we address a critical gap: linking network structures to policy-timed counterfactuals at the jurisdictional scale. While previous works implicitly caution against overclaiming causality, we position our inferred influence edges as directional hypotheses validated through robustness checks and policy simulations.


\section{Problem Definition}

We investigate whether high opioid prescribing rates spread geographically across U.S. states and counties, or whether they emerge independently in each region. Specifically, we address two questions. First, do regions that become high-prescribing tend to be geographically adjacent to regions that were already high-prescribing? If so, this suggests a spatial diffusion process where prescribing norms spread across boundaries. Second, can we identify specific states or counties that consistently preceded others in adopting high prescribing behavior, acting as ``influencers'' or early adopters that may have driven broader regional trends? Answering these questions requires distinguishing between two scenarios: one where prescribing patterns are driven primarily by local factors (demographics, healthcare access, economic conditions) with minimal cross-boundary spillover, and another where a region's prescribing behavior is meaningfully shaped by what happens in neighboring regions. Understanding which scenario better describes the data has direct implications for intervention design, whether to target isolated high-risk areas or to address regional clusters through coordinated multi-state efforts.
\section{Data Collection and Preprocessing}

Opioid dispensing data came from the CDC's public resources. Historical state-level data (2006--2018) were collected by scraping archived pages from the CDC's drug overdose prevention website. For each year, we extracted state names, abbreviations, and rates per 100 persons, mapping abbreviations to FIPS codes using a lookup table. More recent state-level data (2019--2023) came directly from the \href{https://www.cdc.gov/overdose-prevention/data-research/facts-stats/opioid-dispensing-rate-maps.html}{CDC's Opioid Dispensing Rate Maps page}.

County-level data required separate preprocessing. We parsed archived text files (2006--2018) containing county dispensing rates, handling multiple file formats where county-state ordering varied across years. Each record was standardized to include county name, state abbreviation, FIPS code, and dispensing rate. County names were normalized with proper suffixes (County, Parish, Borough) and matched to state identifiers. Missing values (marked as dashes or ``Data unavailable'') were preserved for later handling. Recent county data (2019--2023) were downloaded directly from the CDC, covering approximately 98\% of U.S. counties.

Preprocessing merged historical and recent panels for both geographic levels, standardizing columns and removing national aggregate rows. Text-formatted rates were converted to numeric values. Duplicate state-year and county-year entries were resolved by keeping the most recent record. The final datasets---a state panel (2006--2018) and county panel (2006--2018)---provide the longitudinal prescribing histories needed for influence network construction and spatial modeling.

\section{Methods}

Our approach combines two complementary techniques: (1) influence network construction to identify which regions preceded others in adopting high-prescribing behavior, and (2) spatial autoregressive modeling to predict future prescribing rates and quantify spatial spillover effects. We apply both methods at state and county levels to examine whether spatial dynamics differ across geographic scales.

\subsection{Defining High-Prescribing Status}

The first step is classifying regions as ``high-prescribing'' or not in each year. For state-level analysis, we use the 75th percentile of the national distribution as the threshold $\tau = 87.35$ prescriptions per 100 persons. A state $s$ in year $t$ is classified as high-prescribing if its rate $R_{s,t} > \tau$. This threshold-based approach converts continuous rates into a binary adoption framework suitable for network inference.

For county-level analysis, we use state-specific thresholds rather than a single national cutoff. Within each state, we compute the 75th percentile of county-level rates across all years, then classify counties as high-prescribing relative to their own state's distribution. This relative approach accounts for baseline differences between states—a county with a moderate national rate may still be an outlier within a low-prescribing state.

\subsection{Influence Network Construction}

\textbf{Intuition.} Traditional approaches to studying opioid prescribing treat each region independently, modeling rates as functions of local demographics, healthcare access, or policy variables. This ignores potential cross-boundary effects where prescribing norms in one region influence neighboring regions. Our network-based approach explicitly captures these spatial dependencies by examining temporal precedence: if region $A$ was already high-prescribing when region $B$ transitioned to high status, we hypothesize that $A$ may have influenced $B$. By aggregating these events across years, we build a weighted directed graph that summarizes which regions consistently preceded others.

\textbf{Algorithm.} For each consecutive year pair $(t, t+1)$, we identify:
\begin{itemize}
    \item \textbf{Source set} $S_t$: regions that were high-prescribing in year $t$
    \item \textbf{New adopter set} $N_{t+1}$: regions that were high-prescribing in year $t+1$ but not in year $t$
\end{itemize}

For every source $s \in S_t$ and every new adopter $n \in N_{t+1}$, we create a directed edge $s \rightarrow n$. Edge weights accumulate across years using a time-decay function that gives more weight to early, sustained influence:

\begin{equation}
w(s \rightarrow n) = \sum_{t: s \in S_t, n \in N_{t+1}} \frac{1}{1 + (t - t_s)}
\end{equation}

where $t_s$ is the adoption year of source $s$. This weighting scheme assigns higher influence to regions that adopted early and remained high-prescribing for multiple years, reflecting the intuition that persistent early adopters had more opportunity to influence others than late or transient adopters.

\textbf{State-Level Network.} At the state level, we further constrain edges to geographically adjacent states. We define a neighbor dictionary based on shared borders (e.g., Georgia's neighbors are Alabama, Florida, North Carolina, South Carolina, and Tennessee). This geographic constraint reflects the assumption that influence flows more readily between bordering states through shared healthcare markets, provider networks, and patient populations.

\textbf{County-Level Network.} At the county level, we build separate influence networks within each state. This intra-state approach captures how high-prescribing behavior spreads among counties within the same jurisdiction, avoiding the complexity of cross-state county interactions while still revealing local diffusion patterns.

\subsection{Ranking Influential Regions}

Once the influence network is constructed, we rank nodes by their potential to spread prescribing behavior. We compute weighted out-degree centrality:

\begin{equation}
C_{out}(s) = \sum_{n \in V} w(s \rightarrow n)
\end{equation}

where $V$ is the set of all nodes and $w(s \rightarrow n)$ is the edge weight from $s$ to $n$ (zero if no edge exists). Higher out-degree indicates that a region preceded more adoptions with stronger weights, suggesting greater influence.

We also compute eigenvector centrality, which accounts for the influence of a node's neighbors. A region connected to other highly influential regions receives a higher eigenvector score than one connected only to peripheral nodes. This captures cascade effects where influence propagates through chains of connected regions.

\subsection{Spatial Autoregressive Model}

\textbf{Intuition.} While the influence network reveals historical precedence patterns, it does not directly quantify how much neighbor rates contribute to predicting a region's future rate. To address this, we develop a spatial autoregressive (SAR) model that decomposes prescribing dynamics into self-persistence and spatial spillover components. If spatial spillover is negligible, a region's future rate depends only on its own history. If spillover is substantial, neighbor rates carry independent predictive information.

\textbf{Model Specification.} For each region $s$ and year $t$, we model the next year's prescribing rate as:

\begin{equation}
R_{s,t+1} = \alpha + \beta \cdot R_{s,t} + \gamma \cdot \bar{R}_{N(s),t} + \epsilon_{s,t}
\end{equation}

where:
\begin{itemize}
    \item $R_{s,t}$ is the current prescribing rate (self-history feature)
    \item $\bar{R}_{N(s),t}$ is the average rate of $s$'s geographic neighbors (spatial feature)
    \item $\beta$ is the self-persistence coefficient
    \item $\gamma$ is the spatial spillover coefficient
    \item $\epsilon_{s,t}$ is the error term
\end{itemize}

\textbf{State-Level Implementation.} For states, neighbors are defined by geographic adjacency (shared borders). For isolated states (Alaska, Hawaii), the spatial feature is set to zero. We fit the model using ordinary least squares regression with a temporal train-test split: training on data from 2006--2016 and testing on 2017--2018.

\textbf{County-Level Implementation.} For counties, we use the state-level average rate as the spatial context feature rather than neighboring county rates. This captures how much a county's prescribing behavior is influenced by its broader state environment:

\begin{equation}
R_{c,t+1} = \alpha + \beta \cdot R_{c,t} + \gamma \cdot R_{state(c),t} + \epsilon_{c,t}
\end{equation}

where $R_{state(c),t}$ is the state-level rate for the state containing county $c$. This formulation tests whether counties respond to state-level trends beyond their own historical trajectory.

\textbf{Interpretation of Coefficients.} The coefficient $\beta$ measures self-persistence—how strongly a region's past predicts its future. Values close to 1 indicate high path dependence where rates change slowly. The coefficient $\gamma$ measures spatial spillover—how much neighbor or state-level rates contribute beyond self-history. A positive $\gamma$ supports the hypothesis that prescribing behavior diffuses across boundaries; a near-zero $\gamma$ suggests prescribing is driven primarily by local factors.

\subsection{Superspreader Identification}

At the county level, we identify ``superspreader'' counties that adopted high-prescribing status before their state crossed the threshold. These are counties where:

\begin{equation}
t_{adopt}^{county} < t_{adopt}^{state}
\end{equation}

Superspreaders represent early local hotspots that preceded broader state-level trends. Identifying these counties helps pinpoint where high-prescribing behavior may have originated within each state and which local areas might be priority targets for early intervention.

\subsection{Evaluation Metrics}

We evaluate the spatial autoregressive model using standard regression metrics on the held-out test set:
\begin{itemize}
    \item \textbf{$R^2$ Score}: Proportion of variance in future rates explained by the model
    \item \textbf{Mean Squared Error (MSE)}: Average squared prediction error
\end{itemize}

For influence rankings, we report the top-$k$ most influential regions and examine whether rankings are stable across different centrality measures (out-degree vs. eigenvector). Consistency across metrics increases confidence that identified influencers are robust rather than artifacts of a particular ranking method.




\section{Initial findings and Summary statistics}

The consolidated datasets show clear patterns in opioid prescribing across the United States. The state-level dispensing panel covers 2006–2023 with 918 observations across 51 states, providing an 18-year panel.
A pronounced national decline appears. Early years (2006–2012) saw the highest rates. West Virginia peaked at 145.5 prescriptions per 100 persons in 2008, and several Southern states exceeded 130, including Tennessee (140.0 in 2010), Alabama (136.6 in 2011), and Oklahoma (127.4 in 2012). By 2023, rates had fallen across all states. Hawaii (22.6), California (23.8), and New York (26.3) had the lowest, while Southern states remained highest, though far below earlier peaks—Arkansas (71.5) and Alabama (71.4) were the highest in 2023.

Geographic patterns are clear. Using the CDC threshold of 51.7 prescriptions per 100 persons to define high prescribing, Southern states consistently ranked highest, with Alabama, Arkansas, Mississippi, Louisiana, Tennessee, and Kentucky above the threshold in most early years. Northeastern and Western states generally had lower rates, with Hawaii, California, New Jersey, and New York consistently among the lowest.
Temporal analysis shows the decline accelerated after 2012, aligning with increased policy focus and CDC guidance. Many states transitioned from high to low status during this period. The county-level data reveals within-state variation, with rural and certain metropolitan counties often showing higher rates than state averages.
The policy dataset includes yearly indicators from PDMP implementation and reporting requirements. Most states implemented PDMPs between 1998 and 2017, and the timing and features of these programs varied substantially, allowing analysis of how policy interventions may have influenced prescribing trends.



\section{Future Work Plan}
Future research will extend this state-level opioid dispersion analysis into a multi-scale, real-time surveillance and intervention platform, building on the thorough 8-phase implementation methodology.  In order to capture intra-state variation, especially the urban-rural prescription disparities noted in reviewer feedback, the immediate next phase will involve scaling down to county-level granularity.  After that, provider-level shared patient networks will be integrated to produce hierarchical impact models that span individual prescribers, counties, and states.  Creating dynamic network models that go beyond the present static annual snapshots and reflect time-varying influence strengths and seasonal prescribing patterns will be a crucial step forward.  The approach will be expanded to multi-drug analysis, examining poly-prescribing influence patterns and how prescribing contagion functions across several controlled substances (stimulants, benzodiazepines). Future research will use instrumental variable methodologies for more reliable causal inference and quasi-experimental designs that take advantage of natural policy variances in PDMP rollout time among states in order to address concerns about causation. By integrating real-time early warning systems with the current PDMP infrastructure, the ultimate goal is to move from retrospective analysis to prospective application, giving public health authorities network-informed intervention targeting capabilities. As a result of this evolution, the current research framework will become an operational decision-support tool capable of identifying new hotspots for prescription drug use, allocating intervention resources optimally, and assessing the efficacy of policies through ongoing feedback loops, ultimately leading to more accurate and successful responses to prescription drug epidemics.
\section{Contribution}
This project was done by Vishal Maradana and Rishi Bandi. Vishal focused on data acquisition (scraping, merging) and building the core network construction algorithm. Rishi handled the policy and county data integration, defined the "high-prescribing" adoption threshold, and created the network visualizations. Both members contributed equally to the project's design and this report.
%%
%% The next two lines define the bibliography style to be used, and
%% the bibliography file.
\bibliographystyle{ACM-Reference-Format}
\bibliography{name}


%%
%% If your work has an appendix, this is the place to put it.
\appendix


\end{document}
\endinput
%%%    